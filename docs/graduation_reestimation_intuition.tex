\documentclass[11pt, a4paper]{article}

% --- Packages ---
\usepackage[margin=2.5cm]{geometry}
\usepackage{amsmath, amssymb}
\usepackage{mathtools}
\usepackage{bm}
\usepackage{tcolorbox}
\usepackage{booktabs}
\usepackage{array}
\usepackage{enumitem}
\usepackage{hyperref}
\usepackage{graphicx}

\hypersetup{
  colorlinks=true,
  linkcolor=blue!60!black,
  urlcolor=blue!60!black
}

% --- Three-Box System ---
\newtcolorbox{formaldef}[1][]{
  colback=blue!5!white,
  colframe=blue!60!black,
  fonttitle=\bfseries,
  title={#1},
  sharp corners,
  boxrule=0.8pt
}

\newtcolorbox{intuition}[1][]{
  colback=green!5!white,
  colframe=green!50!black,
  fonttitle=\bfseries,
  title={#1},
  sharp corners,
  boxrule=0.8pt
}

\newtcolorbox{application}[1][]{
  colback=orange!5!white,
  colframe=orange!60!black,
  fonttitle=\bfseries,
  title={#1},
  sharp corners,
  boxrule=0.8pt
}

% --- Custom commands ---
\newcommand{\mx}{m_{x,t}}
\newcommand{\ax}{a_x}
\newcommand{\bx}{b_x}
\newcommand{\kt}{k_t}

% =====================================================================
\begin{document}

\title{\textbf{Why Our Model Couldn't Count Deaths} \\[0.5em]
\large Graduation, Negative $\bx$, and the Re-estimation Paradox \\[0.3em]
\normalsize An Intuitive Guide}
\author{SIMA -- Sistema Integral de Modelaci\'{o}n Actuarial}
\date{\today}

\maketitle

\begin{abstract}
When we applied Lee-Carter to real Mexican mortality data, the standard
re-estimation step failed. This short document explains \emph{why} it failed
using pictures and analogies first, with the math as supporting evidence.
The core story: smoothing mortality rates (graduation) creates a world
where death counts no longer add up, and a few ``rebel'' ages make
the problem unsolvable.
\end{abstract}

\tableofcontents
\newpage

% =====================================================================
\section{The Setup: Two Steps That Don't Mix}
% =====================================================================

Our pipeline has two key steps, each sensible on its own but
incompatible when combined.

\begin{intuition}[Analogy: The Photo and the Census]
Imagine you take a blurry photo of a crowd and use Photoshop to
sharpen it (graduation). Now the faces look cleaner. But if someone
asks ``How many people are in the photo?''\ and you count the sharpened
faces, you get 427. The original blurry photo had 472 faces---some were
hidden behind blur that sharpening removed, and some overlapping faces
got merged.

\medskip
\textbf{Sharpening changed the picture but not reality.} The real
headcount is still 472. Your sharpened model says 427.

\medskip
This is exactly what happens: graduation smooths mortality rates
(sharpens the picture), but the real death count stays the same.
Re-estimation tries to force the sharpened picture to match the
real headcount---and can't.
\end{intuition}

\begin{formaldef}[The Two Steps]
\textbf{Step 1 -- Graduation} (Whittaker-Henderson): Smooth raw rates
$m_{x,t}^{\text{raw}}$ across ages to get $m_{x,t}^{\text{grad}}$.
Minimizes roughness while staying close to data.

\medskip
\textbf{Step 2 -- Re-estimation}: For each year $t$, solve for $k$:
\[
\underbrace{\sum_x E_{x,t} \cdot \exp(\ax + \bx \cdot k)}_{\text{model deaths (graduated world)}}
\;=\;
\underbrace{D_t}_{\text{real deaths (raw world)}}
\]
where $\ax, \bx$ come from graduated data but $D_t$ is the raw
observed count.
\end{formaldef}

\begin{application}[Mexican Data, 1990]
\begin{center}
\begin{tabular}{lr}
\toprule
\textbf{Source} & \textbf{Implied deaths} \\
\midrule
Raw INEGI death count ($D_{1990}$) & 472{,}000 \\
Graduated rates summed ($\sum E_x \cdot m_x^{\text{grad}}$) & $\approx 427{,}000$ \\
\midrule
\textbf{Gap} & $\approx 45{,}000$ (10\%) \\
\bottomrule
\end{tabular}
\end{center}
Re-estimation must bridge this 10\% gap by adjusting $\kt$.
This is already hard. The next section shows why it becomes \emph{impossible}.
\end{application}


% =====================================================================
\section{The Rebel Ages: Negative $\bx$}
% =====================================================================

In Lee-Carter, each age has a sensitivity coefficient $\bx$ that says
\emph{how much that age responds to the general mortality trend}.

\begin{intuition}[Analogy: A Rowing Team]
Think of mortality improvement as a rowing team. The time index $\kt$
is the boat speed. Each age $x$ is a rower:

\begin{itemize}[nosep]
\item \textbf{Positive $\bx$}: This rower pulls \emph{with} the team.
  When $\kt$ drops (boat moves forward), this age's mortality drops too.
\item \textbf{$\bx \approx 0$}: This rower barely contributes. Age
  barely changes regardless of the trend.
\item \textbf{Negative $\bx$}: This rower pulls \emph{backwards}.
  When everyone else improves, this age gets \emph{worse}.
\end{itemize}

\medskip
In our data, ages 77, 78, and 85 are backward rowers.
Out of 101 ages, only 3 rebel---but they break the math.
\end{intuition}

\begin{formaldef}[Why Negative $\bx$ Appears]
Lee-Carter decomposes the residual matrix via SVD. The sign of $\bx$
depends on whether age $x$'s temporal pattern aligns with the dominant
trend. When graduation smooths each year's age profile independently,
it can distort the temporal pattern at specific ages:

\medskip
\begin{center}
\begin{tabular}{lccc}
\toprule
& Year 1990 & Year 2010 & \textbf{Trend} \\
\midrule
Age 76 (graduated) & 0.040 & 0.032 & Improving \\
Age 77 (graduated) & 0.043 & 0.040 & \textbf{Barely changing} \\
Age 78 (graduated) & 0.046 & 0.038 & Improving \\
\bottomrule
\end{tabular}
\end{center}

\medskip
Age 77's stagnation (possibly from noise, age heaping, or cause-of-death
structure) gives it a negative $\bx$ because it moves \emph{opposite}
to the dominant downward trend.
\end{formaldef}

\begin{application}[SIMA: Which Ages Are Rebels?]
After fitting Lee-Carter on graduated Mexican data (1990--2019):

\begin{center}
\begin{tabular}{crl}
\toprule
\textbf{Age} & $\bx$ & \textbf{Likely cause} \\
\midrule
77 & $-0.0041$ & Random noise in time dimension \\
78 & $-0.0023$ & Random noise in time dimension \\
85 & $-0.0059$ & Age heaping (rounding to 85) \\
\bottomrule
\end{tabular}
\end{center}

Graduation smooths across ages (horizontal), but these ages have noise
across \emph{time} (vertical)---a dimension graduation doesn't touch.
\end{application}


% =====================================================================
\section{The U-Shape: Why Root-Finding Fails}
% =====================================================================

The re-estimation equation asks us to find $k$ such that $f(k) = D_t$,
where:
\[
f(k) = \sum_x E_{x,t} \cdot \exp(\ax + \bx \cdot k)
\]

\begin{intuition}[Analogy: The Tug of War]
$f(k)$ is a sum of exponentials. Each age contributes a term
$E_x \cdot \exp(\ax + \bx \cdot k)$:

\begin{itemize}[nosep]
\item Ages with $\bx > 0$ contribute terms that \textbf{grow} as $k$ increases
  (exponential going up-right).
\item Ages with $\bx < 0$ contribute terms that \textbf{shrink} as $k$ increases
  (exponential going down-right).
\end{itemize}

When all $\bx > 0$: every term grows, so $f(k)$ is a monotone escalator
going up forever. Any horizontal target line $D_t$ gets crossed exactly once.

When some $\bx < 0$: the growing and shrinking terms fight.
At extreme negative $k$, the rebels dominate (their exponentials blow up).
At extreme positive $k$, the normal ages dominate.
In between, neither side wins---$f(k)$ dips to a \textbf{valley} (U-shape).

\medskip
If the valley floor is above $D_t$: the curve crosses the target twice
(two solutions exist).

If the valley floor is below $D_t$: \textbf{no crossing, no solution.}
\end{intuition}

\begin{formaldef}[The Math Behind the U-Shape]
The second derivative is always positive:
\[
f''(k) = \sum_x E_{x,t} \cdot \bx^2 \cdot \exp(\ax + \bx \cdot k) > 0
\]
because $\bx^2 \geq 0$. So $f$ is \textbf{always convex} (bowl-shaped).

\medskip
The first derivative:
\[
f'(k) = \sum_x E_{x,t} \cdot \bx \cdot \exp(\ax + \bx \cdot k)
\]
When all $\bx > 0$: $f'(k) > 0$ always (monotone, no U-shape).

When some $\bx < 0$: $f'(k^*) = 0$ at exactly one point $k^*$
(the valley minimum). Whether $f(k^*) \lessgtr D_t$ determines
if solutions exist.
\end{formaldef}

\begin{application}[What Happens in SIMA]

\begin{center}
\renewcommand{\arraystretch}{1.1}
\begin{tabular}{p{6cm}p{6cm}}
\toprule
\textbf{All $\bx > 0$ (ideal case)} & \textbf{Some $\bx < 0$ (our case)} \\
\midrule
$f(k)$ rises monotonically &
$f(k)$ dips then rises (U-shape) \\[0.3em]
Brent's method always finds root &
Valley minimum $\approx 427{,}000$ \\[0.3em]
One unique solution &
Target $D_t = 472{,}000$ is \emph{above} valley \\[0.3em]
 &
$\Rightarrow$ \textbf{No solution exists} \\
\bottomrule
\end{tabular}
\end{center}

The graduation gap (valley at ${\sim}427\text{k}$ vs.\ target at
${\sim}472\text{k}$) combines with the U-shape to make the equation
unsolvable.
\end{application}


% =====================================================================
\section{The Two Problems Reinforce Each Other}
% =====================================================================

\begin{intuition}[The Double Whammy]
Two things go wrong simultaneously, and each makes the other worse:

\begin{enumerate}[nosep]
\item \textbf{Graduation creates the U-shape} (via negative $\bx$):
  The function has a valley instead of being a simple ramp.
\item \textbf{Graduation creates the death-count gap}:
  The target $D_t$ sits above the valley floor.
\end{enumerate}

If only one problem existed, re-estimation would still work:
\begin{itemize}[nosep]
\item U-shape but no gap $\Rightarrow$ valley floor below target $\Rightarrow$ two crossings exist
\item Gap but no U-shape $\Rightarrow$ monotone function always crosses any target
\end{itemize}

\textbf{Both problems together} = valley too high + target too high = no crossing.
Both originate from the same source: graduation.
\end{intuition}


% =====================================================================
\section{Our Solution: Stay in One World}
% =====================================================================

\begin{intuition}[Analogy: Don't Mix Currencies]
Re-estimation fails because it mixes two ``currencies'':

\begin{center}
\begin{tabular}{ll}
\textbf{Left side:} & graduated rates (smoothed pesos) \\
\textbf{Right side:} & raw death counts (real pesos) \\
\end{tabular}
\end{center}

You can't balance an equation where one side is in smoothed pesos
and the other in real pesos. The exchange rate doesn't exist.

\medskip
\textbf{Our fix:} stay entirely in the graduated world.
Use SVD's $\kt$ directly (estimated in log-space from graduated rates).
Everything is internally consistent---all ``smoothed pesos.''
\end{intuition}

\begin{formaldef}[Why SVD $\kt$ Is Consistent]
SVD minimizes:
\[
\sum_{x,t} \left[\ln m_{x,t}^{\text{grad}} - \ax - \bx \cdot \kt \right]^2
\]
All three quantities ($m^{\text{grad}}$, $\ax$, $\kt$) live in the
graduated world. The resulting $\kt$ captures the true mortality trend
with explained variance of 77.7\%.
\end{formaldef}

\begin{application}[Practical Impact]
\begin{center}
\begin{tabular}{lcc}
\toprule
\textbf{Approach} & \textbf{Drift} & \textbf{Result} \\
\midrule
SVD $\kt$ (our choice) & $-1.076$ & Sensible projections \\
Re-estimated $\kt$ (forced) & $-17.6$ & Wildly wrong \\
\bottomrule
\end{tabular}
\end{center}

\medskip
Forcing re-estimation doesn't just fail silently---when the adaptive
bracket search finds approximate roots, the resulting $\kt$ values
are nonsensical. A drift of $-17.6$ would imply mortality halving
every few years, which is biologically impossible.

\medskip
\textbf{The theoretically ``better'' method produces worse results.}
Internal consistency beats theoretical optimality.
\end{application}


% =====================================================================
\section{What About the Accuracy We Lose?}
% =====================================================================

\begin{intuition}[How Much Does It Matter?]
Without re-estimation, model-implied deaths don't match observed deaths
exactly. But for \emph{pricing}, what matters is the \textbf{trend}
of $\kt$, not whether historical deaths match perfectly.

\medskip
Think of it this way: if you're predicting tomorrow's temperature,
it matters more that your thermometer consistently reads 1 degree
high every day (predictable bias, same trend) than that each reading
is exactly correct. The forecast accuracy depends on the slope,
not the intercept.

\medskip
Both SVD and re-estimated $\kt$ follow nearly the same downward
slope. Re-estimation shifts the level but barely changes the
drift. Since premiums depend on projected rates 10--30 years
out, the level shift washes out---the trend dominates.
\end{intuition}

\begin{application}[Quantified Impact on Premiums]
Estimated accuracy loss from skipping re-estimation: $\sim$1--2\% on
premiums. For comparison:
\begin{center}
\begin{tabular}{lr}
\toprule
\textbf{Source of variation} & \textbf{Premium impact} \\
\midrule
COVID inclusion (5 extra years) & +3\% to +10\% \\
Interest rate $\pm$1\% & +8\% to +15\% \\
Skipping re-estimation & $\sim$1--2\% \\
\bottomrule
\end{tabular}
\end{center}
The re-estimation loss is dwarfed by other modeling choices.
\end{application}


% =====================================================================
\section{Summary: The Story in One Page}
% =====================================================================

\begin{center}
\renewcommand{\arraystretch}{1.4}
\begin{tabular}{>{\raggedright}p{3.5cm} p{9cm}}
\toprule
\textbf{Step} & \textbf{What happens} \\
\midrule
Raw data &
INEGI deaths + CONAPO population give noisy $m_{x,t}$ rates. \\

Graduation &
Whittaker-Henderson smooths across ages. Rates become clean,
but total implied deaths drop $\sim$10\%. \\

Lee-Carter SVD &
Decomposes smooth rates into $\ax + \bx \cdot \kt$.
Three ages get negative $\bx$ (noise in time dimension). \\

Re-estimation attempt &
Tries to match graduated model to raw deaths.
Negative $\bx$ creates U-shaped function;
graduation gap puts target above the valley.
\textbf{No solution exists.} \\

Our solution &
Keep SVD $\kt$ (internally consistent).
Skip re-estimation. Lose $\sim$1--2\% accuracy,
gain stable, sensible projections. \\

Alternative &
Poisson log-bilinear model (Brouhns et al., 2002)
avoids the problem entirely by modeling deaths
directly---no SVD, no re-estimation needed. \\
\bottomrule
\end{tabular}
\end{center}

\bigskip

\begin{intuition}[The One-Sentence Summary]
Graduation and re-estimation live in different worlds---one smooths
rates, the other counts deaths---and when three rebel ages create
a U-shaped function, those worlds cannot be reconciled.
\end{intuition}

\end{document}
