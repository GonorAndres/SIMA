\documentclass[11pt,a4paper]{article}

\usepackage[utf8]{inputenc}
\usepackage[T1]{fontenc}
\usepackage{amsmath, amssymb, amsthm}
\usepackage{mathtools}
\usepackage{bm}
\usepackage{tcolorbox}
\usepackage{booktabs}
\usepackage{array}
\usepackage{enumitem}
\usepackage{hyperref}
\usepackage[margin=2.5cm]{geometry}

% Theorem environments
\newtheorem{theorem}{Theorem}[section]
\newtheorem{lemma}[theorem]{Lemma}
\newtheorem{proposition}[theorem]{Proposition}
\newtheorem{corollary}[theorem]{Corollary}
\newtheorem{definition}[theorem]{Definition}
\newtheorem{remark}[theorem]{Remark}

% Three-box system
\newtcolorbox{formaldef}[1][]{
  colback=blue!5!white,
  colframe=blue!60!black,
  fonttitle=\bfseries,
  title={#1},
  sharp corners,
  boxrule=0.8pt
}

\newtcolorbox{intuition}[1][]{
  colback=green!5!white,
  colframe=green!50!black,
  fonttitle=\bfseries,
  title={#1},
  sharp corners,
  boxrule=0.8pt
}

\newtcolorbox{application}[1][]{
  colback=orange!5!white,
  colframe=orange!60!black,
  fonttitle=\bfseries,
  title={#1},
  sharp corners,
  boxrule=0.8pt
}

\newtcolorbox{dimcheck}{
  colback=gray!8!white,
  colframe=gray!50!black,
  fonttitle=\bfseries,
  title={Dimension Check},
  sharp corners,
  boxrule=0.6pt
}

\title{\textbf{SVD, Bilinear Decomposition, and Identifiability Constraints} \\[0.5em]
\large A Bridge Document for SIMA -- Lee-Carter Mortality Modeling}
\author{SIMA Project}
\date{\today}

\begin{document}
\maketitle

\begin{abstract}
This document develops the mathematical foundations connecting Singular Value Decomposition (SVD),
bilinear parameter estimation, and identifiability constraints, culminating in the Lee-Carter
mortality model. Each concept is presented in three layers: formal definition (blue), geometric
intuition (green), and actuarial application (orange). The treatment is at graduate level,
emphasizing the structural reasons why SVD is the natural estimator for bilinear models and
why constraints are mathematically necessary for unique parameterization.
\end{abstract}

\tableofcontents
\newpage

% ============================================================
\section{Notation and Conventions}
% ============================================================

\begin{tabular}{@{} l l l @{}}
\toprule
\textbf{Symbol} & \textbf{Space} & \textbf{Meaning} \\
\midrule
$\bm{A} \in \mathbb{R}^{m \times n}$ & matrix & Arbitrary real matrix \\
$\bm{Z} \in \mathbb{R}^{A \times T}$ & matrix & Lee-Carter residual matrix \\
$\bm{U} \in \mathbb{R}^{m \times m}$ & orthogonal & Left singular vectors \\
$\bm{V} \in \mathbb{R}^{n \times n}$ & orthogonal & Right singular vectors \\
$\bm{\Sigma} \in \mathbb{R}^{m \times n}$ & diagonal & Singular values $\sigma_1 \geq \sigma_2 \geq \cdots \geq 0$ \\
$\bm{u}_i \in \mathbb{R}^{m}$ & column of $\bm{U}$ & $i$-th left singular vector \\
$\bm{v}_i \in \mathbb{R}^{n}$ & column of $\bm{V}$ & $i$-th right singular vector \\
$\sigma_i$ & scalar $\geq 0$ & $i$-th singular value \\
$\|\cdot\|_F$ & norm & Frobenius norm: $\|\bm{A}\|_F = \sqrt{\sum_{ij} a_{ij}^2}$ \\
$\|\cdot\|_2$ & norm & Spectral (operator) norm: $\|\bm{A}\|_2 = \sigma_1(\bm{A})$ \\
$\mathrm{rank}(\bm{A})$ & integer & Number of nonzero singular values \\
$m_{x,t}$ & scalar $> 0$ & Central death rate, age $x$, year $t$ \\
$a_x, b_x, k_t$ & scalars & Lee-Carter parameters \\
\bottomrule
\end{tabular}

\vspace{1em}

Convention: boldface lowercase for vectors ($\bm{u}$), boldface uppercase for matrices ($\bm{A}$),
italic for scalars ($\sigma$). Ages indexed by $x \in \{x_1, \ldots, x_A\}$,
years by $t \in \{t_1, \ldots, t_T\}$.

% ============================================================
\section{Bilinear Forms and the Parameter Product Problem}
% ============================================================

\begin{formaldef}[Definition: Bilinear Parametric Model]
A \emph{bilinear parametric model} for a data matrix $\bm{Z} \in \mathbb{R}^{m \times n}$ is a model of the form:
\[
Z_{ij} = \sum_{r=1}^{R} \beta_i^{(r)} \, \gamma_j^{(r)} + \varepsilon_{ij}
\]
where $\bm{\beta}^{(r)} \in \mathbb{R}^m$ and $\bm{\gamma}^{(r)} \in \mathbb{R}^n$ are unknown parameter vectors,
$R$ is the number of factors, and $\varepsilon_{ij}$ is noise.

In matrix form:
\[
\bm{Z} = \bm{B} \bm{G}^\top + \bm{E}
\]
where $\bm{B} = [\bm{\beta}^{(1)} \cdots \bm{\beta}^{(R)}] \in \mathbb{R}^{m \times R}$
and $\bm{G} = [\bm{\gamma}^{(1)} \cdots \bm{\gamma}^{(R)}] \in \mathbb{R}^{n \times R}$.
\end{formaldef}

\begin{intuition}[Why ``Bilinear''?]
The model is \emph{linear in each factor separately}, but the factors enter as a product.

Fix $\bm{\gamma}$: the model $Z_{ij} = \beta_i \gamma_j$ is linear in $\beta_i$.
Fix $\bm{\beta}$: the model is linear in $\gamma_j$.
Together: \emph{bilinear} --- linear in each, nonlinear jointly.

This is why OLS fails. Ordinary least squares assumes $\bm{y} = \bm{X}\bm{\beta}$ where
$\bm{X}$ is known. Here both $\bm{B}$ and $\bm{G}$ are unknown. The ``design matrix''
is itself a parameter. There is no closed-form normal equation.
\end{intuition}

\begin{application}[Lee-Carter as a Rank-1 Bilinear Model]
The Lee-Carter model:
\[
\ln(m_{x,t}) = a_x + b_x \, k_t + \varepsilon_{x,t}
\]
After centering (subtracting $a_x = \bar{y}_x$), the residual satisfies:
\[
Z_{x,t} = b_x \, k_t + \varepsilon_{x,t}
\]
This is a bilinear model with $R = 1$, $\bm{\beta} = \bm{b} = (b_{x_1}, \ldots, b_{x_A})^\top$,
$\bm{\gamma} = \bm{k} = (k_{t_1}, \ldots, k_{t_T})^\top$.

In matrix form: $\bm{Z} \approx \bm{b}\,\bm{k}^\top$, a rank-1 outer product.
\end{application}

\subsection{The Fundamental Indeterminacy of Bilinear Products}

\begin{formaldef}[Proposition: Scale Indeterminacy]
Let $\bm{Z} = \bm{b}\,\bm{k}^\top$. For any $c \neq 0$, define $\bm{b}^* = c\,\bm{b}$ and
$\bm{k}^* = \bm{k}/c$. Then:
\[
\bm{b}^* (\bm{k}^*)^\top = c\,\bm{b} \cdot \frac{\bm{k}^\top}{c} = \bm{b}\,\bm{k}^\top = \bm{Z}
\]
The factorization is invariant under reciprocal scaling. The data cannot distinguish
$(\bm{b}, \bm{k})$ from $(c\,\bm{b}, \bm{k}/c)$ for any $c \neq 0$.
\end{formaldef}

\begin{formaldef}[Proposition: Location Indeterminacy in Lee-Carter]
In the uncentered model $\ln(m_{x,t}) = a_x + b_x k_t$, define for any constant $d$:
\[
a_x^* = a_x + d \cdot b_x, \qquad k_t^* = k_t - d
\]
Then:
\[
a_x^* + b_x \, k_t^* = (a_x + d\,b_x) + b_x(k_t - d) = a_x + b_x\,k_t
\]
The constant $d$ shifts freely between $a_x$ and $k_t$ without changing the model output.
\end{formaldef}

\begin{intuition}[Geometric Picture of Indeterminacy]
A rank-1 matrix $\bm{b}\,\bm{k}^\top$ defines a \emph{line} in parameter space, not a point.
The line passes through the origin in $(\bm{b}, \bm{k})$-space, parameterized by $c$.

Every point on the line produces the same matrix $\bm{Z}$. Without a constraint to pick
one point on the line, the solution is a one-dimensional family, not unique.

Scale indeterminacy: a line through the origin (choose magnitude).
Location indeterminacy: a plane (choose where $a_x$ ends and $k_t$ begins).
Together: a two-parameter family of equivalent solutions.
\end{intuition}

% ============================================================
\section{Singular Value Decomposition}
% ============================================================

\begin{formaldef}[Theorem: Existence of SVD]
For any $\bm{A} \in \mathbb{R}^{m \times n}$ with $r = \mathrm{rank}(\bm{A})$, there exist:
\begin{itemize}[nosep]
  \item An orthogonal matrix $\bm{U} \in \mathbb{R}^{m \times m}$ (columns $\bm{u}_1, \ldots, \bm{u}_m$)
  \item An orthogonal matrix $\bm{V} \in \mathbb{R}^{n \times n}$ (columns $\bm{v}_1, \ldots, \bm{v}_n$)
  \item Singular values $\sigma_1 \geq \sigma_2 \geq \cdots \geq \sigma_r > 0 = \sigma_{r+1} = \cdots$
\end{itemize}
such that:
\[
\bm{A} = \bm{U}\bm{\Sigma}\bm{V}^\top = \sum_{i=1}^{r} \sigma_i \, \bm{u}_i \, \bm{v}_i^\top
\]
The outer product form shows $\bm{A}$ as a sum of $r$ rank-1 matrices, each weighted by $\sigma_i$.
\end{formaldef}

\begin{dimcheck}
\[
\underbrace{\bm{A}}_{m \times n} = \underbrace{\bm{U}}_{m \times m} \; \underbrace{\bm{\Sigma}}_{m \times n} \; \underbrace{\bm{V}^\top}_{n \times n}
\]
Outer product form: each $\bm{u}_i \bm{v}_i^\top$ is $(m \times 1)(1 \times n) = m \times n$.

In Lee-Carter: $m = A$ (number of ages), $n = T$ (number of years).
\[
\underbrace{\bm{Z}}_{A \times T} = \sum_{i=1}^{r} \sigma_i \; \underbrace{\bm{u}_i}_{A \times 1} \; \underbrace{\bm{v}_i^\top}_{1 \times T}
\]
\end{dimcheck}

\begin{intuition}[SVD as Sequential Best Rank-1 Extraction]
Think of SVD as a greedy algorithm for pattern extraction:

\textbf{Round 1:} Find the single rank-1 matrix $\sigma_1 \bm{u}_1 \bm{v}_1^\top$ that captures
the maximum possible variance (Frobenius norm) of $\bm{A}$. Subtract it from $\bm{A}$.

\textbf{Round 2:} From the residual, find the next best rank-1 matrix $\sigma_2 \bm{u}_2 \bm{v}_2^\top$,
constrained to be orthogonal to the first. Subtract.

\textbf{Round $i$:} Continue until the residual is zero ($i = r$) or negligible.

Each $\sigma_i$ measures ``how much pattern'' was extracted in round $i$.
If $\sigma_1 \gg \sigma_2$, the first round captured almost everything.
\end{intuition}

\subsection{The Eckart-Young-Mirsky Theorem}

\begin{formaldef}[Theorem: Eckart-Young-Mirsky (1936)]
Let $\bm{A} \in \mathbb{R}^{m \times n}$ with SVD $\bm{A} = \sum_{i=1}^{r} \sigma_i \bm{u}_i \bm{v}_i^\top$.
Define the rank-$k$ truncation:
\[
\bm{A}_k = \sum_{i=1}^{k} \sigma_i \, \bm{u}_i \, \bm{v}_i^\top, \qquad k \leq r
\]
Then for any matrix $\bm{B}$ with $\mathrm{rank}(\bm{B}) \leq k$:
\[
\|\bm{A} - \bm{A}_k\|_F \leq \|\bm{A} - \bm{B}\|_F
\]
with equality if and only if $\bm{B} = \bm{A}_k$ (up to components in the null space of the truncation).

The approximation error is:
\[
\|\bm{A} - \bm{A}_k\|_F = \sqrt{\sigma_{k+1}^2 + \sigma_{k+2}^2 + \cdots + \sigma_r^2}
\]
\end{formaldef}

\begin{formaldef}[Proof Sketch (Frobenius norm)]
Since $\bm{U}$ and $\bm{V}$ are orthogonal, the Frobenius norm is invariant under orthogonal transformation:
\[
\|\bm{A} - \bm{B}\|_F^2 = \|\bm{U}^\top(\bm{A} - \bm{B})\bm{V}\|_F^2 = \|\bm{\Sigma} - \bm{U}^\top\bm{B}\bm{V}\|_F^2
\]
Let $\bm{C} = \bm{U}^\top \bm{B} \bm{V}$. Since $\mathrm{rank}(\bm{C}) = \mathrm{rank}(\bm{B}) \leq k$, at least
$r - k$ diagonal entries of $\bm{\Sigma}$ cannot be matched by $\bm{C}$. The minimum of
$\sum_{i} (\sigma_i - C_{ii})^2 + \sum_{i \neq j} C_{ij}^2$ is achieved by setting
$C_{ii} = \sigma_i$ for $i \leq k$ and $C_{ij} = 0$ otherwise, giving the truncated SVD.
\end{formaldef}

\begin{intuition}[Eckart-Young: Optimality Guarantee]
Eckart-Young says: ``You cannot do better than SVD truncation for low-rank approximation.''

It is not merely \emph{a} solution. It is \emph{the} optimal solution.

For $k = 1$ (Lee-Carter): among all possible outer products $\bm{b}\bm{k}^\top$, the one that
minimizes $\|\bm{Z} - \bm{b}\bm{k}^\top\|_F^2$ is exactly $\sigma_1 \bm{u}_1 \bm{v}_1^\top$.

The proportion of variance explained:
\[
\rho = \frac{\sigma_1^2}{\sum_{i=1}^{r} \sigma_i^2} = 1 - \frac{\|\bm{Z} - \bm{A}_1\|_F^2}{\|\bm{Z}\|_F^2}
\]
When $\rho \geq 0.90$, retaining one factor is strongly justified.
\end{intuition}

\begin{application}[Lee-Carter: Rank-1 Optimality]
For the Lee-Carter residual matrix $\bm{Z}$ (ages $\times$ years):
\[
\min_{\bm{b}, \bm{k}} \sum_{x,t} \left(Z_{x,t} - b_x k_t\right)^2 \quad \Longleftrightarrow \quad \min_{\mathrm{rank}(\bm{M})=1} \|\bm{Z} - \bm{M}\|_F^2
\]
The solution is $\bm{M}^* = \sigma_1 \bm{u}_1 \bm{v}_1^\top$, and we set:
\begin{align*}
b_x^{\text{raw}} &= u_{1,x} \quad \text{(first left singular vector, age component)} \\
k_t^{\text{raw}} &= \sigma_1 \cdot v_{1,t} \quad \text{(scaled first right singular vector, time component)}
\end{align*}
This is the least-squares bilinear estimator. SVD is not an ad hoc choice ---
it is the unique solution to the optimization problem.
\end{application}

% ============================================================
\section{Identifiability Constraints}
% ============================================================

\begin{formaldef}[Definition: Identifiability]
A parametric model $f(\bm{\theta})$ is \emph{identifiable} if distinct parameter values produce
distinct model outputs:
\[
f(\bm{\theta}_1) = f(\bm{\theta}_2) \implies \bm{\theta}_1 = \bm{\theta}_2
\]
A model that is not identifiable has an \emph{equivalence class} of parameters that all
produce the same fit. The set of equivalent parameters forms a manifold in parameter space.
\end{formaldef}

\begin{formaldef}[Theorem: Lee-Carter Indeterminacy is Two-Dimensional]
The Lee-Carter model $\ln(m_{x,t}) = a_x + b_x k_t$ has a two-parameter family of equivalent
solutions. Given any solution $(a_x, b_x, k_t)$, the full equivalence class is:
\[
\mathcal{E} = \left\{ (a_x + d \cdot c \, b_x, \; c \, b_x, \; k_t/c - d) \;\middle|\; c \neq 0, \; d \in \mathbb{R} \right\}
\]
\textbf{Proof.} Substituting:
\begin{align*}
a_x^* + b_x^* k_t^* &= (a_x + dc\,b_x) + (c\,b_x)(k_t/c - d) \\
&= a_x + dc\,b_x + b_x k_t - dc\,b_x \\
&= a_x + b_x k_t
\end{align*}
The parameter $c$ controls scale (rescaling $\bm{b}$ and $\bm{k}$ inversely).
The parameter $d$ controls location (shifting between $\bm{a}$ and $\bm{k}$).
Two free parameters, hence a two-dimensional equivalence class.
\end{formaldef}

\begin{formaldef}[Lee-Carter Normalization Constraints]
To select a unique representative from $\mathcal{E}$, impose:
\begin{align}
\sum_{x} b_x &= 1 \tag{C1: scale constraint} \\
\sum_{t} k_t &= 0 \tag{C2: location constraint}
\end{align}
\textbf{C1} fixes $c$: if $\sum_x b_x^{\text{raw}} = s$, then $c = 1/s$ gives $\sum_x c\,b_x^{\text{raw}} = 1$.

\textbf{C2} fixes $d$: if $\bar{k} = \frac{1}{T}\sum_t k_t$, then $d = \bar{k}$ gives $\sum_t (k_t - d) = 0$, with compensation $a_x \mapsto a_x + b_x \bar{k}$.

Two constraints for two degrees of freedom yields a unique solution.
\end{formaldef}

\begin{intuition}[Constraints as Coordinate Choices]
Think of the equivalence class as an infinite line of equivalent parameter sets.
Each constraint slices away one degree of freedom:

$\sum b_x = 1$ chooses a specific ``length'' for $\bm{b}$, pinning a point on the scale axis.

$\sum k_t = 0$ chooses a specific ``origin'' for the time index, pinning a point on the location axis.

Together they select exactly one point from the two-dimensional equivalence manifold.

The choice is conventional, not unique. Alternative valid constraints exist
(e.g., $\sum b_x^2 = 1$, or $k_{t_1} = 0$). Different constraints give different parameter
values but identical model output. The Lee-Carter convention ($\sum b_x = 1$, $\sum k_t = 0$)
is chosen for interpretability: $b_x$ become proportions and $a_x$ become true time-averages.
\end{intuition}

\begin{application}[Implementation: Constraint Enforcement]
After SVD gives raw estimates:
\begin{enumerate}[nosep]
\item $b_x^{\text{raw}} = u_{1,x}$, \quad $k_t^{\text{raw}} = \sigma_1 \cdot v_{1,t}$
\item $s = \sum_x b_x^{\text{raw}}$
\item $b_x = b_x^{\text{raw}} / s$, \quad $k_t = k_t^{\text{raw}} \cdot s$ \hfill (C1 enforced)
\item $\bar{k} = \frac{1}{T}\sum_t k_t$
\item $k_t \leftarrow k_t - \bar{k}$, \quad $a_x \leftarrow a_x + b_x \cdot \bar{k}$ \hfill (C2 enforced)
\end{enumerate}
Note: step 5 modifies $a_x$. After this, $a_x$ is no longer the simple row mean of $\ln(m_{x,t})$
--- it has absorbed the mean level of $k_t$.
\end{application}

% ============================================================
\section{Variance Decomposition and the One-Factor Justification}
% ============================================================

\begin{formaldef}[Variance Partition via Singular Values]
The total variance (sum of squared entries) of the residual matrix $\bm{Z}$ satisfies:
\[
\|\bm{Z}\|_F^2 = \sum_{i=1}^{r} \sigma_i^2
\]
The proportion attributed to factor $i$ is:
\[
\rho_i = \frac{\sigma_i^2}{\sum_{j=1}^{r} \sigma_j^2}
\]
For Lee-Carter, one retains only factor 1. The unexplained proportion is $1 - \rho_1 = \sum_{i \geq 2} \rho_i$.
\end{formaldef}

\begin{intuition}[Why Does One Factor Suffice for Mortality?]
Mortality improvement across ages is driven by shared macro-level forces:
advances in medicine, sanitation, nutrition, public health infrastructure.

If each age improved independently (age 40 from antibiotics, age 70 from cardiac surgery,
with no correlation), the residual matrix would have several large singular values.
No single factor would dominate.

But empirically, $\rho_1 \geq 0.90$ for virtually all national populations. This reflects a
biological and sociological reality: the forces that reduce mortality are broad, systemic,
and correlated across ages. One underlying process drives most of the change.

When this assumption breaks (HIV epidemic affecting ages 20--40 disproportionately,
or infant mortality declining while elderly mortality stagnates), extended models with
$R = 2$ or more factors (Renshaw-Haberman, Cairns-Blake-Dowd) become necessary.
\end{intuition}

% ============================================================
\section{From SVD to the Full Lee-Carter Procedure}
% ============================================================

\begin{application}[Complete Estimation Pipeline]
\textbf{Input:} Matrix $\bm{M} \in \mathbb{R}^{A \times T}$ of observed central death rates $m_{x,t} > 0$.

\begin{enumerate}[nosep]
\item \textbf{Log transform:} $Y_{x,t} = \ln(m_{x,t})$
\item \textbf{Estimate} $\hat{a}_x$: $\hat{a}_x = \frac{1}{T}\sum_{t=1}^{T} Y_{x,t}$
\item \textbf{Compute residuals:} $Z_{x,t} = Y_{x,t} - \hat{a}_x$
\item \textbf{SVD:} $\bm{Z} = \bm{U}\bm{\Sigma}\bm{V}^\top$
\item \textbf{Extract rank-1:} $b_x^{\text{raw}} = u_{1,x}$, \; $k_t^{\text{raw}} = \sigma_1 \cdot v_{1,t}$
\item \textbf{Normalize:} enforce $\sum b_x = 1$, $\sum k_t = 0$ (adjust $a_x$)
\item \textbf{Re-estimate $k_t$} (optional but recommended): solve
\[
\sum_x d_{x,t} = \sum_x L_{x,t} \cdot \exp(\hat{a}_x + \hat{b}_x \, k_t)
\]
for each $t$, so that total predicted deaths match observed deaths.
\item \textbf{Forecast $k_t$}: fit ARIMA (typically random walk with drift) to $\{k_t\}$
\item \textbf{Project:} $\ln(\hat{m}_{x,t^*}) = \hat{a}_x + \hat{b}_x \, \hat{k}_{t^*}$ for future $t^*$
\item \textbf{Convert:} $\hat{q}_{x,t^*} = 1 - \exp(-\hat{m}_{x,t^*})$
\end{enumerate}
\end{application}

\begin{dimcheck}
Full dimension trace through the pipeline:
\begin{align*}
\bm{M} &\in \mathbb{R}^{A \times T}_{>0}  & \text{(raw rates)} \\
\bm{Y} &\in \mathbb{R}^{A \times T}        & \text{(log rates)} \\
\hat{\bm{a}} &\in \mathbb{R}^{A}           & \text{(one per age)} \\
\bm{Z} &\in \mathbb{R}^{A \times T}        & \text{(residuals, same shape as } \bm{Y}\text{)} \\
\bm{U} &\in \mathbb{R}^{A \times A}        & \text{(age singular vectors)} \\
\bm{\Sigma} &\in \mathbb{R}^{A \times T}   & \text{(diagonal, } \min(A,T) \text{ values)} \\
\bm{V} &\in \mathbb{R}^{T \times T}        & \text{(year singular vectors)} \\
\hat{\bm{b}} &\in \mathbb{R}^{A}           & \text{(one per age, from } \bm{u}_1\text{)} \\
\hat{\bm{k}} &\in \mathbb{R}^{T}           & \text{(one per year, from } \sigma_1 \bm{v}_1\text{)} \\
\hat{\bm{b}} \, \hat{\bm{k}}^\top &\in \mathbb{R}^{A \times T} & \text{(rank-1 approximation of } \bm{Z}\text{)}
\end{align*}
\end{dimcheck}

% ============================================================
\section{The Observable: Why $m_x$, Not $q_x$}
% ============================================================

\begin{formaldef}[Definition: Central Death Rate]
\[
m_{x,t} = \frac{d_{x,t}}{L_{x,t}}
\]
where $d_{x,t}$ is the count of deaths at age $x$ in year $t$, and $L_{x,t}$ is the person-years
of exposure at age $x$ in year $t$.

Unit: deaths per person-year. Domain: $(0, +\infty)$.
\end{formaldef}

\begin{formaldef}[Conversion to Probability]
Under the constant force of mortality assumption ($\mu_{x+s} = m_x$ for $0 \leq s < 1$):
\[
q_x = 1 - e^{-m_x}
\]
Under the Uniform Distribution of Deaths (UDD) assumption:
\[
q_x = \frac{m_x}{1 + \frac{1}{2} m_x}
\]
For $m_x < 0.1$ both yield $q_x \approx m_x$ with relative error $< 5\%$.
\end{formaldef}

\begin{intuition}[The Observation Hierarchy]
\begin{center}
\begin{tabular}{@{} c c c @{}}
\toprule
\textbf{Level} & \textbf{Quantity} & \textbf{Status} \\
\midrule
1 & Deaths $d_{x,t}$ and exposure $L_{x,t}$ & Raw data (certificates, census) \\
2 & $m_{x,t} = d_{x,t} / L_{x,t}$ & Direct calculation \\
3 & $q_{x,t} = f(m_{x,t})$ & Requires distributional assumption \\
4 & $l_x, d_x, D_x, N_x, \ldots$ & Requires full life table construction \\
\bottomrule
\end{tabular}
\end{center}
Each level adds assumptions. Lee-Carter models at level 2: the closest to raw data
that still produces a clean mathematical object. This minimizes modeling risk and
maximizes connection to what is actually observed.
\end{intuition}

\begin{application}[Transformation Guarantees]
The chain $\ln(m_x) \xrightarrow{\exp} m_x \xrightarrow{1 - e^{-(\cdot)}} q_x$ provides:
\begin{itemize}[nosep]
  \item $\ln(m_x) \in \mathbb{R}$: unconstrained for SVD, ARIMA, any linear method
  \item $m_x = e^{\ln(m_x)} > 0$: positivity guaranteed
  \item $q_x = 1 - e^{-m_x} \in (0, 1)$: valid probability guaranteed
\end{itemize}
No post-processing, clipping, or projection needed. The transformation chain
is structurally safe for any output of the Lee-Carter model.
\end{application}

% ============================================================
\section{Summary Table}
% ============================================================

\begin{center}
\small
\renewcommand{\arraystretch}{1.3}
\begin{tabular}{@{} p{3.2cm} p{4.5cm} p{2.5cm} p{3.5cm} @{}}
\toprule
\textbf{Operation} & \textbf{Mathematics} & \textbf{Dimensions} & \textbf{Purpose} \\
\midrule
Log transform & $Y_{x,t} = \ln(m_{x,t})$ & $A \times T$ & Unconstrained space \\
Row centering & $Z_{x,t} = Y_{x,t} - \bar{Y}_x$ & $A \times T$ & Extract $a_x$, isolate trend \\
SVD & $\bm{Z} = \bm{U}\bm{\Sigma}\bm{V}^\top$ & $A\!\times\!A$, diag, $T\!\times\!T$ & Optimal decomposition \\
Rank-1 truncation & $\bm{Z} \approx \sigma_1 \bm{u}_1 \bm{v}_1^\top$ & $A \times T$ & Single-factor approx. \\
Extract $b_x$ & $b_x = u_{1,x} / \sum u_{1,x}$ & $A \times 1$ & Age sensitivity \\
Extract $k_t$ & $k_t = \sigma_1 v_{1,t} \cdot \sum u_{1,x}$ & $1 \times T$ & Time index \\
Normalize & $\sum b_x = 1$, $\sum k_t = 0$ & --- & Unique identification \\
Forecast $k_t$ & $k_{t+1} = k_t + \delta + \varepsilon$ & scalar series & Project mortality \\
Back-transform & $m_x = e^{a_x + b_x k_t}$ & $A \times T$ & Recover rates \\
Convert & $q_x = 1 - e^{-m_x}$ & $A \times T$ & Probability for life table \\
\bottomrule
\end{tabular}
\end{center}

% ============================================================
\section{Key Identities and Properties}
% ============================================================

\begin{enumerate}[nosep]
\item \textbf{SVD existence:} Every real matrix has an SVD (not unique if singular values coincide).
\item \textbf{Eckart-Young:} $\bm{A}_k = \arg\min_{\mathrm{rank}(\bm{B}) \leq k} \|\bm{A} - \bm{B}\|_F$.
\item \textbf{Frobenius norm via singular values:} $\|\bm{A}\|_F^2 = \sum \sigma_i^2$.
\item \textbf{Orthogonal invariance:} $\|\bm{Q}\bm{A}\bm{R}\|_F = \|\bm{A}\|_F$ for orthogonal $\bm{Q}, \bm{R}$.
\item \textbf{Bilinear indeterminacy:} $\bm{b}\bm{k}^\top = (c\bm{b})(\bm{k}/c)^\top$ for all $c \neq 0$.
\item \textbf{Constraint count = indeterminacy dimension:} 2 constraints for 2 free parameters.
\item \textbf{Positivity guarantee:} $\exp(\cdot): \mathbb{R} \to \mathbb{R}_{>0}$ and $1 - e^{-x}: \mathbb{R}_{>0} \to (0,1)$.
\end{enumerate}

% ============================================================
\section{What This Document Does Not Cover (Yet)}
% ============================================================

\begin{itemize}[nosep]
\item \textbf{$k_t$ re-estimation:} Newton-Raphson adjustment to match observed death counts (Section~5 step~7).
\item \textbf{ARIMA forecasting of $k_t$:} random walk with drift, confidence intervals, fan charts.
\item \textbf{Poisson GLM alternative:} maximum likelihood estimation under Poisson death counts
      (Brouhns, Denuit, Vermunt, 2002), which relaxes the equal-variance assumption of SVD.
\item \textbf{Multi-factor extensions:} Renshaw-Haberman (cohort effect), CBD (logit of $q_x$),
      Age-Period-Cohort models.
\item \textbf{Whittaker-Henderson graduation:} smoothing raw $m_x$ before Lee-Carter estimation.
\item \textbf{Mexican regulatory context:} CNSF/LISF requirements for mortality projection.
\end{itemize}

\end{document}
